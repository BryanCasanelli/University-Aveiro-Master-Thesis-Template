\clearpage
\section{Template Information and Instructions}
This LaTeX version was implemented in January 2025, based on the university's official Word template.

\subsection{Page Setup}
Following the original template specifications:
\begin{itemize}
    \item Paper size: A4
    \item First pages margins:
    \begin{itemize}
        \item Left: 1.50cm
        \item Right: 1.50cm
        \item Top: 0.80cm
        \item Bottom: 1.00cm
    \end{itemize}
    \item Main content margins:
    \begin{itemize}
        \item Left: 3cm
        \item Right: 2.5cm
        \item Top: 3cm
        \item Bottom: 3cm
    \end{itemize}
    \item Page numbers are positioned at the bottom right of each page.
\end{itemize}

\subsection{Directory Structure}
The template follows this structure:
\begin{verbatim}
thesis/
|-- abstract/           # Abstracts and keywords
|   |-- abstract_en.tex # English abstract
|   |-- abstract_pt.tex # Portuguese abstract
|   |-- keywords_en.tex # English keywords
|   +-- keywords_pt.tex # Portuguese keywords
|-- img/               # Images folder
|-- sections/          # Content sections
|-- single_pages/      # Initial pages
|-- acronyms.tex      # Acronyms definitions
|-- main.tex          # Main document
+-- references.bib    # Bibliography
\end{verbatim}

\subsection{Fonts}
The template uses the following free fonts:
\begin{itemize}
    \item First pages use \textbf{TeX Gyre Heros}, a free alternative to Helvetica. While the original template specifies Helvetica, it is a proprietary font that each user would need to purchase. TeX Gyre Heros, based on Nimbus Sans, provides a visually similar open-source alternative.
    \item Main content uses \textbf{Linux Libertine}, a free and elegant font family. According to the specifications, the body font is not restricted, so users can change it to their preferred font. Linux Libertine was chosen as the default for its professional appearance, excellent readability, and comprehensive character set, including extensive math support.
\end{itemize}

\subsection{Adding Content}
\textbf{To add abstracts and keywords:}
\begin{enumerate}
    \item Edit the corresponding files in \texttt{abstract/}:
    \begin{itemize}
        \item \texttt{abstract\_en.tex} for English abstract
        \item \texttt{abstract\_pt.tex} for Portuguese abstract
        \item \texttt{keywords\_en.tex} for English keywords
        \item \texttt{keywords\_pt.tex} for Portuguese keywords
    \end{itemize}
\end{enumerate}

\textbf{To add new sections:}
\begin{enumerate}
    \item Create a new .tex file in \texttt{sections/}
    \item Add it using:
    \begin{verbatim}
    \input{sections/your_file}
    \end{verbatim}
\end{enumerate}

\textbf{To add acronyms:}
\begin{enumerate}
    \item Open \texttt{acronyms.tex}
    \item Add new entries using:
    \begin{verbatim}
    \newacronym{abc}{ABC}{Acronym Basic Component}
    \end{verbatim}
    \item While the traditional \verb|\gls{acronym}| command is available, this template provides several enhanced commands for better flexibility:
    \begin{itemize}
        \item \verb|\longshort{acronym}| displays the full form followed by the abbreviation in parentheses. After first use, only shows the full form.
        \item \verb|\shortlong{acronym}| displays the abbreviation followed by the full form in parentheses. After first use, only shows the abbreviation.
        \item \verb|\forcelongshort{acronym}| forces the long-short format regardless of previous usage.
        \item \verb|\forceshortlong{acronym}| forces the short-long format regardless of previous usage.
    \end{itemize}
    \item Note that for \verb|\gls|, \verb|\longshort|, and \verb|\shortlong|, the complete form will only be shown on first use. Subsequent uses will display according to their respective styles.
    \item By default (command \verb|\gls|), the template uses the short-long style (abbreviation followed by full form). This can be modified in \texttt{main.tex} by changing:
    \begin{verbatim}
    \setabbreviationstyle[acronym]{short-long}
    \end{verbatim}
\end{enumerate}

\textbf{To add bibliography entries:}
\begin{enumerate}
    \item Open \texttt{references.bib}
    \item Add entries in BibTeX format:
    \begin{verbatim}
    @article{reference-key,
        author  = {Author Name},
        title   = {Title},
        journal = {Journal},
        year    = {2024},
    }
    \end{verbatim}
    \item To cite in the text, use \verb|\cite{key}|.
\end{enumerate}

\subsection{Features}
\begin{itemize}
    \item \textbf{Automatic Index Generation:}
    \begin{itemize}
        \item Table of contents
        \item List of figures
        \item List of tables
        \item List of acronyms
    \end{itemize}
    
    \item \textbf{Bibliography Management:}
    \begin{itemize}
        \item Default citation style: numeric-comp
        \item To change the citation style, modify the biblatex style parameter in \texttt{main.tex}:
        \begin{verbatim}
\usepackage[style=your-style]{biblatex}
        \end{verbatim}
        where \texttt{your-style} can be ieee, apa, chicago, etc.
    \end{itemize}
    
    \item \textbf{Scientific Area Colors:}
    \begin{itemize}
        \item Template includes all Pantone colors defined in the original specifications.
        \item Change the active color in \texttt{main.tex} using:
        \begin{verbatim}
\colorlet{pantonecolor}{your_color}
        \end{verbatim}
        where \texttt{your\_color} can be: artscomm, sciences, education, accounting1, accounting2, economics, engineering, languages, or health.
    \end{itemize}
\end{itemize}

\subsection{Note}
This implementation was originally designed for engineering theses; minor adjustments might be needed for other fields.